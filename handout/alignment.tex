% Define the top matter
\setModuleTitle{Read Alignment}
\setModuleAuthors{%
  Sonika Tyagi\mailto{sonika.tyagi@agrf.org.au}
  Gayle Philip \mailto{unimelb.edu.au}
}
\setModuleContributions{%
  Mathieu Bourgey \mailto{mathieu.bourgey@mcgill.ca}%
}

%  Start: Module Title Page
\chapter{\moduleTitle}
\newpage
% End: Module Title Page

\section{Key Learning Outcomes}

After completing this practical the trainee should be able to:
\begin{itemize}
  \item Perform the simple NGS data alignment task against one interested reference data
  \item Learn about the SAM/BAM formats for further manipulation 
  \item Be able to sort and index BAM format for visualisation purposes.
\end{itemize}

\section{Resources You'll be Using}
 
\subsection{Tools Used}
\begin{description}[style=multiline,labelindent=0cm,align=left,leftmargin=0.5cm]
  \item[BWA Burrow Wheel Algorithm]\hfill\\
  	\url{http://bio-bwa.sourceforge.net}
  \item[Samtools]\hfill\\
  	\url{http://picard.sourceforge.net/}
\end{description}

\section{Useful Links}
 
\begin{description}[style=multiline,labelindent=0cm,align=left,leftmargin=0.5cm]
  \item[SAM Specification]\hfill\\
    \url{http://samtools.sourceforge.net/SAM1.pdf}
  \item[Explain SAM Flags]\hfill\\
    \url{http://picard.sourceforge.net/explain-flags.html}
\end{description}

\subsection{Sources of Data}
% TODO more specific about how the data was derived for this module
  \url{http://sra.dnanexus.com/studies/ERP001071}

\clearpage

\section{Introduction}

\begin{information}
The goal of this hands-on session is to perform an NGS alignment on the sequencing data coming from a tomour and normal group of samples. We will align raw sequencing data to the mouse genome using BWA aligner and then we will discuss at the the sequence alignment and mapping format (SAM). SAM to BAM coversion, indexing and sorting will also be demostrated. These are important and essential steps for downstreaming processing of the aligned BAM files. 
 
This data is the whole genome sequencing of a lung adenocarcinoma patient AK55. It was downloaded from \texttt{ERP001071} (a link is provided in the data source section). Only the HiSeq2000 data for \texttt{Blood} and \texttt{liverMets} were analysed.

Accession numbers associated with read data assigned by the European Bioinformatics Institute (EBI) start with 'ER'. e.g. ERP is the study and ERR is the run. The original FASTQ files downloaded had the ERR number in front of each read name in the FASTQ file. The read name had to be edited to remove the ERR number at the start of the name. This had caused problems for downstream programs such as Picard for marking optical duplicates.

We have used 4 Blood samples (8 PE \texttt{.fastq.gz} files) and 5 Liver samples (10 PE \texttt{.fastq.gz} files) data from this study to perform the whole genome alignment using the BWA aligner. The whole process took >150K CPU seconds per sample and the precomputed alignment will be used in different sections of this workshop. 
\end{information}

\section{Prepare the Environment}

\begin{information}
By now you know about the raw sequence fastq format generated by the Illumina sequencers. Next we will see how fastq files are aligned with the reference genome and what are the resulting standard alignment format. 
In the interest of time we have selected only 1 Million paired reads from a sample Blood to demonstrate a BWA command. The remainining alignments have already been performed for you and will be required in the subsequent modules of the workshop.

The input data for this section can be found in the \texttt{alignment}
directory on your desktop. Please follow the commands below to go to the right folder and view top 10 lines of the input fastq file:
\end{information}

\begin{steps}
Open the Terminal.

First, go to the right folder, where the data are stored.
\begin{lstlisting}
cd ~/alignment
ls
head -10 input/SM_Blood*_R1.fastq
\end{lstlisting}

\end{steps}

\section{Alignment}

\begin{information}
You already know that there are a number of competing tools for short read alignment, each with its own set of strengths, weaknesses, and caveats. Here we will use BWA, a widely used aligner based on Burrow Wheel Algorithm.  The alignment involves two steps- first, indexing the genome and then running the alignment command.  
BWA is a software package for mapping low-divergent sequences against a large reference genome, such as the human genome. It consists of three algorithms: BWA-backtrack, BWA-SW and BWA-MEM. The first algorithm is designed for Illumina sequence reads up to 100bp, while the rest two for longer sequences ranged from 70bp to 1Mbp. BWA-MEM and BWA-SW share similar features such as long-read support and split alignment, but BWA-MEM, which is the latest, is generally recommended for high-quality queries as it is faster and more accurate. BWA-MEM also has better performance than BWA-backtrack for 70-100bp Illumina reads (ref:bwa manual).
\end{information}

\begin{steps}
BWA has a number of parameters in order to perform the alignment. To view them all type

\begin{lstlisting}
bwa <press enter> 
\end{lstlisting}

BWA uses indexed genome for the alignment in order to keep its memory footprint small. Because of time constraints we will not be running the indexing command. It is run only once for a version of genome, the complete command to index the human genome version hg19 is given below. 

\begin{warning}
  You DO NOT need to run this command. This has already been run for you.
  \begin{lstlisting}
  bwa index -p bwaIndex/human_g1k_v37.fasta -a bwtsw human_g1k_v37.fasta
  \end{lstlisting}
\end{warning}

This command will output 6 files that constitute the index. These files that have the prefix \texttt{human_v37} are stored in the \texttt{bwa\_index} subdirectory. To view if they files have been successfully created type:

We have used the following arguments for the indexing of the genome.
\begin{description}[style=multiline,labelindent=0cm,align=right,leftmargin=\descriptionlabelspace,rightmargin=1.5cm,font=\ttfamily]
  \item[-p] = Prefix of the output database [same as db filename] 
  \item[-a] = Algorithm for constructing BWT index. This method works with the whole human genome.
  \item[reference genome file name] = the last argument is the name of the reference genome file in the fasta format
\end{description}

The FASTQ files we are working with are Sanger encoded (Phred+33), which is the
\begin{lstlisting}
ls -l bwaIndex
\end{lstlisting}
\end{steps}

\begin{information}
Now that the genome is indexed we can move on to the actual alignment. The first
argument for \texttt{bwa} is the basename of the index for the genome to be searched;
in our case this is \texttt{human_v37}.

\end{information}

\begin{steps}
Align the reads from Blood samples using the following command: 

\begin{lstlisting}
bwa mem -M -t 4 -R @RG\tSM:Blood\tID:ERR059354\tLB:lb\tPL:ILLUMINA human_g1k_v37 Blood_ID_ERR059354_R1.fastq.gz Blood_ID_ERR059354_R2.fastq.gz > Blood_ID_ERR059354.sam
\end{lstlisting}

The above command outputs the alignment in SAM format and stores them in the
file \texttt{Blood_ID_ERR059354.sam}.
\end{steps}

\begin{note}
We have used the following arguments for the alignment of the reads.
\begin{description}[style=multiline,labelindent=0cm,align=right,leftmargin=\descriptionlabelspace,rightmargin=1.5cm,font=\ttfamily]
  \item[mem] = fast mode of high quality input such the Illumina
  \item[-M] = flags extra hits as secondary. This is needed for compatibility with other tools downstream.
  \item[-t] = Number of threads.
  \item[-R] = Complete read group header line.
\end{description}

\end{note}

\begin{steps}
Look at the top 10 lines of the SAM file by typing:

\begin{lstlisting}
head -n 10 SM_Blood*.sam
\end{lstlisting}
\end{steps}

\begin{questions}
Can you distinguish between the header of the SAM format and the actual alignments?
\begin{answer}
The header line starts with the letter `@', i.e.: 

% replace this with the actual output
\begin{tabular}{lllll}
@HD & VN:1.0 & SO:unsorted & & \\
@SQ & SN:chr1 & LN:195471971 & & \\
@PG & ID:Bowtie2 &  PN:bowtie2   & VN:2.2.4  & CL:``/tools/bowtie2/bowtie2-default/bowtie2-align-s --wrapper basic-0 -x bowtie\_index/mm10 -q Oct4.fastq'' \\
\end{tabular}

While, the actual alignments start with read id, i.e.:

% replace this with the actual output
\begin{tabular}{llll}
SRR002012.45 & 0 & etc & \\
SRR002012.48 & 16 & chr1 & etc \\
\end{tabular}
\end{answer}

What kind of information does the header provide?
\begin{answer}
\begin{itemize}
  \item @HD: Header line; VN: Format version; SO: the sort order of alignments.
  \item @SQ: Reference sequence information; SN: reference sequence name; LN: reference sequence length.
  \item @PG: Read group information; ID: Read group identifier; VN: Program version; CL: the command line that produces the alignment.
\end{itemize}
\end{answer}

% update the answers
To which chromosome are the reads mapped? 
\begin{answer}
Chromosome 1.
\end{answer}
\end{questions}

\section{Manipulate SAM output}

\begin{note}
SAM files are rather big and when dealing with a high volume of NGS data,
storage space can become an issue. As we have already seen, we can convert SAM
to BAM files (their binary equivalent that are not human readable) that occupy
much less space.
\end{note}

\begin{steps}
Convert SAM to BAM using \texttt{samtools view} and store the output in the file
\texttt{Blood.bam}. You have to instruct \texttt{samtools view} that the input is in SAM
format (\texttt{-S}), the output should be in BAM format (\texttt{-b}) and that
you want the output to be stored in the file specified by the \texttt{-o}
option:

\begin{lstlisting}
samtools view -bSo SM_Blood.bam SM_Blood.sam
\end{lstlisting}
\end{steps}

\begin{advanced}
Compute summary stats for the Flag values associated with the alignments using:

\begin{lstlisting}
samtools flagstat SM_Blood.bam
\end{lstlisting}
\end{advanced}

\section{Post alignment processing}

\begin{information}
When uploading a BAM file into the genome browser, such as IGV, the browser will look for the
index of the BAM file in the same folder where the BAM files is. The index file
should have the same name as the BAM file and the suffix \texttt{.bai}. Finally, to
create the index of a BAM file you need to make sure that the file is sorted
according to chromosomal coordinates.
\begin{note}
IGV is a stand-alone genome browser. Please check their website
(\url{http://www.broadinstitute.org/igv/}) for all the formats that IGV
can display. 
\end{note}

\end{information}


\begin{steps}
Sort alignments according to chromosomal position and store the result in the
file with the prefix \texttt{Blood.sorted}:

\begin{lstlisting}
samtools sort Blood.bam Blood.sorted
\end{lstlisting}

Index the sorted file.

\begin{lstlisting}
samtools index Blood.sorted.bam
\end{lstlisting}

The indexing will create a file called \texttt{Blood.sorted.bam.bai}. Note that
you don't have to specify the name of the index file when running
\texttt{samtools index}, it simply appends a \texttt{.bai} suffix to the input
BAM file.
\end{steps}


\begin{questions}
What is the 
\begin{answer}
The 
\end{answer}
\end{questions}



